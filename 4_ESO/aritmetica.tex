%-------------------------------------------------------------------------------
% Aritmetica.
%-------------------------------------------------------------------------------

\section{Conjuntos numéricos}

\begin{ejer}
Indica a qué conjuntos numéricos pertenecen los siguientes números: 


\end{ejer}

\begin{table}[!h]
%\caption{Desglose de presupuesto del proyecto.}
\begin{center}
\begin{tabularx}{\textwidth}{|X | X | X | X | X | X|} 
 \hline
 \textbf{Número} & \textbf{$\mathbb{N}$} & \textbf{$\mathbb{Z}$} & \textbf{$\mathbb{Q}$} & \textbf{$\mathbb{I}$} & \textbf{$\mathbb{R}$} \\ 
 \hline
 $-2.01$ & & & & & \\
 \hline
  $\sqrt[3]{-4}$ & & & & & \\
 \hline
   $0.121212$... & & & & & \\
 \hline
    $\sqrt[3]{-1000}$ & & & & & \\
 \hline
   $1.223334$... & & & & & \\
 \hline
    $\sqrt{-4}$ & & & & & \\
 \hline
    $\frac{1}{2}$ & & & & & \\
 \hline
\end{tabularx}
%\label{tabla_pres}
\end{center}
\end{table}

\begin{ejer}
Escribe en forma de fracción las siguientes expresiones decimales exactas y redúcelas, comprueba con la calculadora que está bien: \\
a) 7.92835 \\ b) 291.291835 \\ c) 0.23
\end{ejer}

\begin{ejer}
Escribe en forma de fracción las siguientes expresiones decimales periódicas, redúcelas y comprueba que está bien: \\
a) 2.353535... \\ b) 87.2365656565... \\ c) 0.9999... \\ d) 26.5735735735...
\end{ejer}

\section{Operaciones: Potencias, raíces y notación científica}

\subsection{Potencias}

\begin{ejer}
Calcula las siguientes potencias: \\
a) $-3^3$ \\ b) $(2+1)^3$ \\ c) $-(-2x)^2$
\end{ejer}

\begin{ejer}
Calcula las siguientes operaciones con potencias: \\
a) $3^5\cdot 9^2$ \\
b) $(2^3)^3$ \\
c) $\dfrac{5^3}{5^0}$ \\
d) $\dfrac{3^4}{3^{-5}}$ \\
e) $(x+1)\cdot (x+1)^3$ \\
f) $\dfrac{(x+2)^3}{(x+2)^4}$ \\
g) $(x+3) \cdot (x+3)^{-3}$
\end{ejer}

\subsection{Raíces}

\begin{ejer}
Saca factores fuera de la raíz $\sqrt{108}$.
\end{ejer}

\begin{ejer}
Escribe los siguiente radicales como una sola raíz: $\dfrac{\sqrt{3}\cdot \sqrt[3]{4}}{\sqrt[6]{24}}$.
\end{ejer}

\begin{ejer}
Calcula: \\
a) $\left( \sqrt[12]{(x+1)^3} \right) $ \\
b) $\sqrt{\sqrt[4]{\frac{x}{5y}}} / \sqrt{ \sqrt[4]{\frac{3x}{y^2}}}$
\end{ejer}

\begin{ejer}
Calcula y simplifica, racionalizando si es necesario: \\
a) $\dfrac{\sqrt[4]{x^3y^3} \cdot \sqrt[3]{x^4y^5}}{\sqrt[6]{x^5y^4}}$ \\
b) $\dfrac{5\sqrt{5}-2\sqrt{2}}{\sqrt{5}-2}$
\end{ejer}

\begin{ejer}
Expresa como potencias: \\
a) $\sqrt{3}$ \\
b) $\sqrt[3]{27}$ \\
c) $\sqrt[15]{27}$
\end{ejer}

\subsection{Notación científica}

\begin{ejer}
Calcula: \\
a) $(7.83\cdot 10^{5})\cdot (1.84 \cdot 10^{13})$ \\
b) $(5.2 \cdot 10^{-4}) / (3.2 \cdot 10^{-6})$ \\
c) $\dfrac{3 \cdot 10^{-5} +7 \cdot 10^{-4}}{10^{6}-5\cdot \cdot 10^{5}}$ \\
d) $\dfrac{7.35\cdot 10^{4}}{5 \cdot 10^{-3}} + 3.2 \cdot 10^{7}$
\end{ejer}

\section{Logaritmos}

\begin{ejer}
Calcula: \\
a) $log_3 81$ \\
b) $log_2 128$ \\
c) $log_3 \sqrt{243}$ \\
\end{ejer}

\begin{ejer}
Calcula aplicando las propiedades: \\
a) $log_5 \left( \dfrac{x^2}{y^5z} \right)^3$ \\
b) $ln \sqrt[5]{\dfrac{4x^2}{e^3}}$ \\
c) $log \left( \dfrac{a^3 b^2}{c^4 d} \right) $
\end{ejer}

\section{Intervalos}

\begin{ejer}
Expresa como intervalo o semirrecta, en forma de conjunto (usando desigualdades) y representa gráficamente: \\
a) Porcentaje superior al 26 \%. \\
b) Edad inferior o igual a 18 años. \\
c) Números cuyo cubo sea superior a 8. \\
d) Números positivos cuya parte entera tiene 3 cifras. \\
e) Temperatura inferior a 25 ºC. \\
f) Números para los que existe su raíz cuadrada (es un número real). \\
g) Números que estén de 5 a una distancia inferior a 4.
\end{ejer}

\section{Razones, proporciones y porcentajes}

\subsection{Porcentajes}

\begin{ejer}
Expresa en tanto por ciento las siguientes proporciones: \\
a) $\dfrac{27}{100}$ \\
b) 1 de cada 2. \\
c) $\dfrac{52}{90}$
\end{ejer}

\begin{ejer}
Si sabemos que los alumnos rubios de una clase son el 16 \% y hay 4 alumnos rubios, ¿cuántos alumnos hay en total?
\end{ejer}

\begin{ejer}
Un depósito de 2 000 litros de capacidad contiene en este momento 1 036 litros. ¿Qué tanto por ciento representa?
\end{ejer}

\begin{ejer}
La proporción de los alumnos de una clase de 4º de ESO que han aprobado Matemáticas fue del 70 \%. Sabiendo que en la clase hay 30 alumnos, ¿cuántos han suspendido?
\end{ejer}

\begin{ejer}
Una fábrica ha pasado de tener 130 obreros a tener 90. Expresa la disminución en porcentaje.
\end{ejer}

\begin{ejer}
Calcula el precio final de un lavavajillas que costaba 520 € más un 21 \% de IVA,al que se le ha aplicado un descuento sobre el coste total del 18 \%.
\end{ejer}

\begin{ejer}
Copia en tu cuaderno y completa:
a) De una factura de 1340 € he pagado 1200 €. Me han aplicado un ........ \% de descuento \\
b) Me han descontado el 9 \% de una factura de ....... € y he pagado 280 €. \\
c) Por pagar al contado un mueble me han descontado el 20 \% y me he ahorrado 100 €. ¿Cuál era el precio del mueble sin descuento?
\end{ejer}

\begin{ejer}
El precio inicial de un electrodoméstico era 500 euros. Primero subió un 10 \% y después bajó un 30 \%. ¿Cuál es su precio actual? ¿Cuál es el porcentaje de incremento o descuento?
\end{ejer}

\begin{ejer}
Una persona ha comprado acciones de bolsa en el mes de enero por un valor de 10 000 €. De enero a febrero estas acciones han aumentado un 8 \%, pero en el mes de febrero han disminuido un 16 \% ¿Cuál es su valor a finales de febrero? ¿En qué porcentaje han aumentado o disminuido?
\end{ejer}

\begin{ejer}
El precio inicial de una enciclopedia era de 300 € y a lo largo del tiempo ha sufrido variaciones. Subió un 10 \%, luego un 25 \% y después bajó un 30 \%. ¿Cuál es su precio actual? Calcula la variación porcentual.
\end{ejer}

\begin{ejer}
En una tienda de venta por Internet se anuncian rebajas del 25 \%, pero luego cargan en la factura un 20 \% de gastos de envío. ¿Cuál es el porcentaje de incremento o descuento? ¿Cuánto tendremos que pagar por un artículo que costaba 30 euros? ¿Cuánto costaba un artículo por el que hemos pagado 36 euros?
\end{ejer}

\subsection{Razones y proporcionalidad}

\begin{ejer}
Copia en tu cuaderno y completa la tabla de proporción directa. Calcula la razón de proporcionalidad. Representa gráficamente los puntos. Determina la ecuación de la recta.
\end{ejer}

\begin{table}[!h]
\begin{center}
\begin{tabularx}{\textwidth}{|X | X | X | X | X | X| X|} 
 \hline
 \textbf{Litros} & 12 & 7.82 &  & 1 & & 50 \\
 \hline
 \textbf{Euros} & 36 &  & 9.27 &  & 10 & \\
 \hline 
\end{tabularx}
\end{center}
\end{table}

\begin{ejer}
Calcula los términos que faltan para completar las proporciones: \\
a) $\frac{24}{100} = \frac{30}{x}$ \\
b) $\frac{x}{80}=\frac{46}{12}$ \\
c) $\frac{3.6}{12.8} = \frac{x}{60}$
\end{ejer}

\begin{ejer}
Para embaldosar un recinto, 7 obreros han dedicado 80 horas de trabajo. Completa en tu cuaderno la siguiente tabla y determina la constante de proporcionalidad..
\end{ejer}

\begin{table}[!h]
\begin{center}
\begin{tabularx}{\textwidth}{|X | c | c | c | c | c| c | c|} 
 \hline
 \textbf{Número de obreros} & 1 & 5 & 7 & 12 &  &  & 60 \\
 \hline
 \textbf{Horas de trabajo} &  &  & 80 &  & 28 & 10 & \\
 \hline 
\end{tabularx}
\end{center}
\end{table}

\begin{ejer}
En una receta nos dicen que para hacer una mermelada de frutas del bosque necesitamos un kilogramo de azúcar por cada dos kilogramos de fruta. Queremos hacer 7 kilogramos de mermelada, ¿cuántos kilogramos de azúcar y cuántos de fruta debemos poner?
\end{ejer}

\begin{ejer}
Dos pantalones nos costaron 28 €, ¿cuánto pagaremos por 7 pantalones?
\end{ejer}

\begin{ejer}
Al cortar una cantidad de madera hemos conseguido 5 paneles de 1.25 m de largo. ¿Cuántos paneles conseguiremos si ahora tienen 3 m de largo?
\end{ejer}

\begin{ejer}
La altura de una torre es proporcional a su sombra (a una misma hora). Una torre que mide 12 m tiene una sombra de 25 m. ¿Qué altura tendrá otra torre cuya sombra mida 43 m?
\end{ejer}

\begin{ejer}
En un huerto ecológico se utilizan 5 000 kg de un tipo de abono de origen animal que se sabe que tiene un 12 \% de nitratos. Se cambia  el tipo de abono, que ahora tiene un 15 \% de nitratos, ¿cuántos kilogramos se necesitarán del nuevo abono para que las plantas reciban la misma cantidad de nitratos?
\end{ejer}

\begin{ejer}
Para envasar cierta cantidad de leche se necesitan 8 recipientes de 100 litros de capacidad cada uno. Queremos envasar la misma cantidad de leche empleando 20 recipientes. ¿Cuál deberá ser la capacidad de esos recipientes?
\end{ejer}

\begin{ejer}
Cinco personas comparten lotería, con 10, 6, 12, 7 y 5 participaciones respectivamente. Si han obtenido un premio de 18 000 € ¿Cuánto corresponde a cada uno?
\end{ejer}

\begin{ejer}
Tres socios han invertido 20 000 €, 34 000 € y 51 000 € este año en su empresa. Si los beneficios a repartir a final de año ascienden a 31 500€, ¿cuánto corresponde a cada uno?
\end{ejer}

\begin{ejer}
En un concurso se acumula puntuación de forma inversamente proporcional al número de errores. Los cuatro finalistas, con 10, 5, 2 y 1 error, deben repartirse los 2 500 puntos. ¿Cuántos puntos
recibirá cada uno?
\end{ejer}

