%-------------------------------------------------------------------------------
% Aritmetica.
%-------------------------------------------------------------------------------

\section{Conjuntos numéricos}

\begin{ejer}
Indica a qué conjuntos numéricos pertenecen los siguientes números: 


\end{ejer}

\begin{table}[!h]
%\caption{Desglose de presupuesto del proyecto.}
\begin{center}
\begin{tabularx}{\textwidth}{|X | X | X | X | X | X|} 
 \hline
 \textbf{Número} & \textbf{$\mathbb{N}$} & \textbf{$\mathbb{Z}$} & \textbf{$\mathbb{Q}$} & \textbf{$\mathbb{I}$} & \textbf{$\mathbb{R}$} \\ 
 \hline
 $-2.01$ & & & & & \\
 \hline
  $\sqrt[3]{-4}$ & & & & & \\
 \hline
   $0.121212$... & & & & & \\
 \hline
    $\sqrt[3]{-1000}$ & & & & & \\
 \hline
   $1.223334$... & & & & & \\
 \hline
    $\sqrt{-4}$ & & & & & \\
 \hline
    $\frac{1}{2}$ & & & & & \\
 \hline
\end{tabularx}
%\label{tabla_pres}
\end{center}
\end{table}

\begin{ejer}
Escribe en forma de fracción las siguientes expresiones decimales exactas y redúcelas, comprueba con la calculadora que está bien: \\
a) 7.92835 \\ b) 291.291835 \\ c) 0.23
\end{ejer}

\begin{ejer}
Escribe en forma de fracción las siguientes expresiones decimales periódicas, redúcelas y comprueba que está bien: \\
a) 2.353535... \\ b) 87.2365656565... \\ c) 0.9999... \\ d) 26.5735735735...
\end{ejer}

\section{Operaciones: Potencias, raíces y notación científica}

\subsection{Potencias}

\begin{ejer}
Calcula las siguientes potencias: \\
a) $-3^3$ \\ b) $(2+1)^3$ \\ c) $-(-2x)^2$
\end{ejer}

\begin{ejer}
Calcula las siguientes operaciones con potencias: \\
a) $3^5\cdot 9^2$ \\
b) $(2^3)^3$ \\
c) $\dfrac{5^3}{5^0}$ \\
d) $\dfrac{3^4}{3^{-5}}$ \\
e) $(x+1)\cdot (x+1)^3$ \\
f) $\dfrac{(x+2)^3}{(x+2)^4}$ \\
g) $(x+3) \cdot (x+3)^{-3}$
\end{ejer}

\subsection{Raíces}

\begin{ejer}
Saca factores fuera de la raíz $\sqrt{108}$.
\end{ejer}

\begin{ejer}
Escribe los siguiente radicales como una sola raíz: $\dfrac{\sqrt{3}\cdot \sqrt[3]{4}}{\sqrt[6]{24}}$.
\end{ejer}

\begin{ejer}
Calcula: \\
a) $\left( \sqrt[12]{(x+1)^3} \right) $ \\
b) $\sqrt{\sqrt[4]{\frac{x}{5y}}} / \sqrt{ \sqrt[4]{\frac{3x}{y^2}}}$
\end{ejer}

\begin{ejer}
Calcula y simplifica, racionalizando si es necesario: \\
a) $\dfrac{\sqrt[4]{x^3y^3} \cdot \sqrt[3]{x^4y^5}}{\sqrt[6]{x^5y^4}}$ \\
b) $\dfrac{5\sqrt{5}-2\sqrt{2}}{\sqrt{5}-2}$
\end{ejer}

\begin{ejer}
Expresa como potencias: \\
a) $\sqrt{3}$ \\
b) $\sqrt[3]{27}$ \\
c) $\sqrt[15]{27}$
\end{ejer}

\subsection{Notación científica}

\begin{ejer}
Calcula: \\
a) $(7.83\cdot 10^{5})\cdot (1.84 \cdot 10^{13})$ \\
b) $(5.2 \cdot 10^{-4}) / (3.2 \cdot 10^{-6})$ \\
c) $\dfrac{3 \cdot 10^{-5} +7 \cdot 10^{-4}}{10^{6}-5\cdot \cdot 10^{5}}$ \\
d) $\dfrac{7.35\cdot 10^{4}}{5 \cdot 10^{-3}} + 3.2 \cdot 10^{7}$
\end{ejer}

\section{Intervalos}

\begin{ejer}
Expresa como intervalo o semirrecta, en forma de conjunto (usando desigualdades) y representa gráficamente: \\
a) Porcentaje superior al 26 \%. \\
b) Edad inferior o igual a 18 años. \\
c) Números cuyo cubo sea superior a 8. \\
d) Números positivos cuya parte entera tiene 3 cifras. \\
e) Temperatura inferior a 25 ºC. \\
f) Números para los que existe su raíz cuadrada (es un número real). \\
g) Números que estén de 5 a una distancia inferior a 4.
\end{ejer}

\section{Valor absoluto}



\section{Razones, proporciones y porcentajes}

