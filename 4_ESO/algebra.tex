%-------------------------------------------------------------------------------
% Aritmetica.
%-------------------------------------------------------------------------------

\section{Polinomios y fracciones algebraicas}

\begin{ejer}
Escribe en lenguaje algebraico los siguientes enunciados, referidos a dos números cualesquiera x e y: \\
a) La mitad del opuesto de su suma. \\
b) La suma de sus cubos. \\
c) El cubo de su suma. \\
d) El inverso de su suma. \\
e) La suma de sus inversos.
\end{ejer}

\begin{ejer}
Calcula el valor numérico de las siguientes expresiones algebraicas para el valor o valores que se indican: \\
a) $x^2 + 7x - 12$ para $x = 0$. \\
b) $(a + b)^2 - (a^2 + b^2)$ para $a = -3$ y $b = 4$. \\
c) $a^2 - 5a + 2$ para $a = -1$.
\end{ejer}

\begin{ejer}
Efectúa los siguientes cálculos: \\
a) $x^2\cdot (-3x^2-3x+1)\cdot 2x^3 +x^3-4x^2+5$ \\
b) $(2x-3)\cdot (-3x^2-5x+4) -(x+2)\cdot (x+1)^2$ \\
c) $\dfrac{2x^3-x^2-x+7}{x^2-2x+4}$ \\
d) $\dfrac{-6x^5+x^2+1}{x^2+1}$ \\
\end{ejer}

\begin{ejer}
Efectúa los siguientes cálculos: \\
a) $\dfrac{2x+1}{x^2+1}+\dfrac{3}{x}$ \\
b) $\dfrac{x-2}{x^2+3x} / \dfrac{x-2}{x+3}$
\end{ejer}

\begin{ejer}
Simplifica las siguientes fracciones algebraicas:\\
a) $\dfrac{3x^2-6x}{9x^2+15}$ \\
b) $\dfrac{a^3-5a^2}{7a^3+4a^2}$ \\
c) $\dfrac{x^2y+3xy^2}{4xy}$ \\
d) $\dfrac{2a^2b^2+3ab}{a^3b-ab}$
\end{ejer}

\begin{ejer}
Factoriza los siguientes polinomios mediante la regla de Ruffini:\\
a) $2x^3+3x^2-11x-6$ \\
b) $3x^2+4x+1$
\end{ejer}

\section{Ecuaciones e inecuaciones con racionales}

\subsection{Ecuaciones}

\subsubsection{Ecuaciones de primer grado}

\begin{ejer}

\end{ejer}

\begin{ejer}

\end{ejer}

\begin{ejer}

\end{ejer}

\begin{ejer}

\end{ejer}

\begin{ejer}

\end{ejer}

\begin{ejer}

\end{ejer}

\section{Sistemas de Ecuaciones Lineales con 2 incógnitas}





