%-------------------------------------------------------------------------------
% Aritmetica.
%-------------------------------------------------------------------------------

\section{Polinomios y fracciones algebraicas}

\begin{ejer}
Escribe en lenguaje algebraico los siguientes enunciados, referidos a dos números cualesquiera x e y: \\
a) La mitad del opuesto de su suma. \\
b) La suma de sus cubos. \\
c) El cubo de su suma. \\
d) El inverso de su suma. \\
e) La suma de sus inversos.
\end{ejer}

\begin{ejer}
Calcula el valor numérico de las siguientes expresiones algebraicas para el valor o valores que se indican: \\
a) $x^2 + 7x - 12$ para $x = 0$. \\
b) $(a + b)^2 - (a^2 + b^2)$ para $a = -3$ y $b = 4$. \\
c) $a^2 - 5a + 2$ para $a = -1$.
\end{ejer}

\begin{ejer}
Efectúa los siguientes cálculos: \\
a) $x^2\cdot (-3x^2-3x+1)\cdot 2x^3 +x^3-4x^2+5$ \\
b) $(2x-3)\cdot (-3x^2-5x+4) -(x+2)\cdot (x+1)^2$ \\
c) $\dfrac{2x^3-x^2-x+7}{x^2-2x+4}$ \\
d) $\dfrac{-6x^5+x^2+1}{x^2+1}$ \\
\end{ejer}

\begin{ejer}
Efectúa los siguientes cálculos: \\
a) $\dfrac{2x+1}{x^2+1}+\dfrac{3}{x}$ \\
b) $\dfrac{x-2}{x^2+3x} / \dfrac{x-2}{x+3}$
\end{ejer}

\begin{ejer}
Simplifica las siguientes fracciones algebraicas:\\
a) $\dfrac{3x^2-6x}{9x^2+15}$ \\
b) $\dfrac{a^3-5a^2}{7a^3+4a^2}$ \\
c) $\dfrac{x^2y+3xy^2}{4xy}$ \\
d) $\dfrac{2a^2b^2+3ab}{a^3b-ab}$
\end{ejer}

\begin{ejer}
Factoriza los siguientes polinomios mediante la regla de Ruffini:\\
a) $2x^3+3x^2-11x-6$ \\
b) $3x^2+4x+1$
\end{ejer}

\section{Ecuaciones e inecuaciones con racionales}

\subsection{Ecuaciones}

\subsubsection{Ecuaciones de primer grado}

\begin{ejer}
Resuelve las siguientes ecuaciones: \\
a) $(x+3)-2(x-3)=2x+3$ \\
b) $\dfrac{5x+1}{6}=\dfrac{4x-2}{9}$ \\
c) $\dfrac{5x+7}{2}-\dfrac{2x+4}{3} = \dfrac{3x+9}{4}+5$ \\
d) $\dfrac{2x+1}{x}=\dfrac{2}{3}$
\end{ejer}

\begin{ejer}
Dos hermanos tienen 11 y 9 años, y su madre 35. Halla el número de años que han de pasar para que la edad de la madre sea igual a la suma de las edades de los hijos.
\end{ejer}

\begin{ejer}
En una ferretería se venden tornillos en cajas de tres tamaños: pequeña, mediana y grande. La caja grande contiene el doble que la mediana y la mediana 25 tornillos más que la pequeña. He comprado una caja de cada tamaño y en total hay 375 tornillos, ¿cuántos tornillos hay en cada caja?
\end{ejer}

\begin{ejer}
En un avión viajan 330 pasajeros de tres países: españoles, alemanes y franceses. Hay 30 franceses más que alemanes y de españoles hay el doble que de franceses y alemanes juntos. ¿Cuántos hay de cada país?
\end{ejer}

\begin{ejer}
Tenemos tres peceras y 56 peces. Los tamaños de las peceras son pequeño, mediano y grande, siendo la pequeña la mitad de la mediana y la grande el doble. Como no tenemos ninguna preferencia en cuanto al reparto de los peces, decidimos que en cada una de ellas haya una cantidad de peces proporcional al tamaño de cada pecera. ¿Cuántos peces pondremos en cada pecera?
\end{ejer}

\begin{ejer}
Un árbol de 7 m de altura es alcanzado por un rayo y lo parte a cierta altura del suelo. La punta del trozo roto cae a 3 m de la base del árbol, formando un triángulo con el otro trozo del árbol. ¿A qué altura se rompió?
\end{ejer}


\subsubsection{Ecuaciones de segundo grado}

\begin{ejer}
Resuelve las siguientes ecuaciones: \\
a) $x^2-x-6=0$ \\
b) $2x^2-7x+3=0$ \\
c) $x^2+6x+9=0$ \\
d) $2x^2=7x-3$
\end{ejer}

\begin{ejer}
Resuelve las siguientes ecuaciones: \\
a) $x^2-49 = 0$ \\
b) $x^2+x=0$ \\
c) $x^2-65=-1$ \\
d) $x^2-3x+2x^2=-9x$ \\
\end{ejer}

\begin{ejer}
En un triángulo rectángulo, el lado mayor es 3 cm más largo que el mediano, el cual, a su vez, es 3 cm más largo que el pequeño. ¿Cuánto miden los lados?
\end{ejer}

\begin{ejer}
Uno de los lados de un rectángulo mide 6 cm más que el otro. ¿Cuáles son las dimensiones si su área es 91 $cm^2$?
\end{ejer}

\begin{ejer}
Si se aumenta en 2 cm la longitud de cada una de las aristas de un cubo, el volumen del mismo aumenta 218 $cm^3$. Calcula la longitud de la arista
\end{ejer}

\begin{ejer}
En un triángulo rectángulo, el lado mayor es 3 cm más largo que el mediano, el cual, a su vez, es 3 cm más largo que el pequeño. ¿Cuánto miden los lados?
\end{ejer}

\subsubsection{Ecuaciones bicuadradas}

\begin{ejer}
Resuelve las siguientes ecuaciones: \\
a) $x^4-10x^2+9=0$ \\
b) $x^4-13x^2+36=0$ \\
c) $x^4-61x^2+900=0$ \\
d) $x^4-25x^2=-144$ \\
\end{ejer}

\begin{ejer}
Resuelve las siguientes ecuaciones: \\
a) $x^4+x^2=0$ \\
b) $x^4-65=-1$ \\
c) $x^^4-3x^2+2x^4=-9x^2$
\end{ejer}

\subsubsection{Ecuaciones racionales}

\begin{ejer}
Resuelve las siguientes ecuaciones: \\
a) $\dfrac{1}{x^2-x}-\dfrac{1}{x-1}=0$ \\
b) $\dfrac{1}{x-2}+\dfrac{1}{x+2}=\dfrac{1}{x^2-4}$ \\
c) $\dfrac{3}{x}=1+\dfrac{x-13}{6}$
\end{ejer}

\begin{ejer}
 Halla un número entero sabiendo que la suma con su inverso es 26/5.
\end{ejer}

\begin{ejer}
Dos caños A y B llenan juntos una piscina en dos horas, A lo hace por sí solo en tres horas menos que B. ¿Cuántas horas tarda a cada uno separadamente?
\end{ejer}

\begin{ejer}
Un caño tarda dos horas más que otro en llenar un depósito y abriendo los dos juntos se llena en 1 hora y 20 minutos. ¿Cuánto tiempo tardará en llenarlo cada uno por separado?
\end{ejer}

\subsubsection{Ecuaciones radicales}

\begin{ejer}

\end{ejer}

\begin{ejer}

\end{ejer}

\subsubsection{De todos}

\begin{ejer}
Resuelve: \\
a) $(2+x)^2-3+5x-x^2=9x+1$ \\
b) $3\cdot (2-x)-\dfrac{x+3}{2}=5x+\dfrac{x}{2}$ \\
c) 
\end{ejer}

\subsection{Inecuaciones}

\begin{ejer}

\end{ejer}

\section{Sistemas de Ecuaciones Lineales con 2 incógnitas}

\begin{ejer}
¿Cuántos litros de leche con un 10\% de grasa hemos de mezclar con otra leche que tiene un 4\% de grasa para obtener 18 litros de leche con un 6\% de grasa?
\end{ejer}




